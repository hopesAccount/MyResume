\resheading{项目经历}
  \begin{itemize}[leftmargin=*]
    \item
      \ressubsingleline{极化敏感共形阵列的单脉冲测向方法研究}{算法理论预研}{2020.02 -- 2020.07}
      {\small
      \begin{itemize}
        \item 项目简介: 针对非规则偶极子布阵的极化敏感共形阵列,展开单脉冲测向方法的研究。
        \item 项目职责:负责算法的理论可行性验证,以及算法的仿真实现。
        \item 职责业绩: 给出了一些算法的理论测向误差;利用最大似然测向方法实现了双极化通道联合测向,克服了单通道测向对极化辅助角敏感的问题。
      \end{itemize}
        }
    \item
      \ressubsingleline{一种高精度多点线性约束的自适应单脉冲测向方法}{算法改进}{2019.09 -- 2019.12}
      {\small
      \begin{itemize}
        \item 项目简介: 对三点约束自适应单脉冲测向方法的缺陷(远离约束点处的测向误差较大)进行改进。
        \item 项目职责:负责算法的理论分析、可行性验证以及算法效果对比的仿真实现。
        \item 职责业绩: 改善了原三点约束方法的测角精度,最大测角误差降低了 80\%(最大误差从 0.6 度降至 0.1 度)。
      \end{itemize}
      }
    \item
    \ressubsingleline{共形阵列的单脉冲测向研究}{算法理论预研}{2019.05 -- 2019.07}
    {\small
    \begin{itemize}
      \item 项目属性: 中电十四所合作项目
      \item 项目简介: 针对各个阵元带有单位方向信息(非全向)的共形阵列,展开单脉冲测向方法研究。
      \item 项目职责: 负责整个共形阵项目的单脉冲测向部分,对测向算法的理论可行性进行验证,并对比各个方法在该阵列结构下的测向误差。
      \item 职责业绩: 实现了非全向天线共形阵列的单脉冲测向;在阵列视轴方向 4 度的邻域内,最大测角误差不超过 1 度。
    \end{itemize}
    }
    \item 
    \ressubsingleline{MPRPC 分布式网络通信框架}{C++}{2020.06 -- 2020.08}
    {\small
    \begin{itemize}
      \item 项目简介: 基于 protobuf 和 muduo 的 RPC 分布式网络通信框架的开发
      \item 项目职责: 使用 protobuf 作为消息序列化和反序列化的工具,将 RPC 请求和 RPC 返回值序列化为二进制数据,
      并在服务发布端和服务调用端进行反序列化解码;服务端网络使用高并发的 muduo 库进行设计,
      而客户端直接使用 linux 的系统调用实现连接请求和数据传输;使用 zookeeper 作为 RPC 服务的注册中心,存储 RPC 服务的节点信息。
      \item 职责业绩: 封装 RPC 的调用过程,向 RPC 的服务发布方和服务调用方提供接口,
      使调用方能够透明的调用远程服务提供方的远程方法,而不用显式的区分本地调用和远程调用。
      \item 项目仓库: https://github.com/metaCoder-00/MpRpc
    \end{itemize}
    }
    \item
    \ressubsingleline{EHK-40 血沉检测仪}{C}{2018.02 -- 2018.06}
    {\small
    \begin{itemize}
      % \item 项目属性:教育部高等学校自动化类专业教学指导委员会的全国性竞赛。
      \item 项目简介: 医用血沉检测仪器的研发。
      \item 项目职责: 参与 EHK-40 血沉检测仪的研发,主要负责血沉测量和测量数据的记录与发送功能模块,
      同时负责和项目甲方实时的沟通联络。
      \item 职责业绩: 使用 C 语言程序设计,设计嵌入式系统,实现了 40 个通道独立测量血沉值,并将测量值保存至本机,
      当有需要时便于查看,并且可以将测量数据发送至医院的粒子系统,实现数据共享。
    \end{itemize}
    }
    \item
    \ressubsingleline{2017年瑞萨杯-全国大学生电子设计竞赛}{自动控制类比赛}{2017.08 -- 2017.08}
    {\small
    \begin{itemize}
      % \item 项目属性:教育部高等学校自动化类专业教学指导委员会的全国性竞赛。
      \item 项目简介: 利用摄像头识别环境信息,完成系统的控制功能。
      \item 项目职责: 参与 2017 年全国大学生电子设计竞赛的控制组,负责控制系统中的图像处理和识别模块,
      使用 C 语言程序设计,设计嵌入式系统,完成了图像区域分割的功能,将识别后的信息传给控制模块。
      \item 职责业绩: 完成了基本功能,识别分割后的区域。
    \end{itemize}
    }
    \item
    \ressubsingleline{第十一届恩智浦杯智能车大赛}{自动控制类比赛}{2016.02 -- 2016.07}
    {\small
    \begin{itemize}
      % \item 项目属性:教育部高等学校自动化类专业教学指导委员会的全国性竞赛。
      \item 项目简介: 利用摄像头识别赛道信息,完成智能小车的自动循迹功能。
      \item 项目职责: 使用 C 语言程序设计,设计嵌入式系统,依据摄像头识别模块给出的信息,完成自动控制功能。
      \item 职责业绩: 取得了华北赛区三等奖的成绩。
    \end{itemize}
    }
  \end{itemize}